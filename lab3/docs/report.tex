\documentclass[12pt]{article}

\usepackage{fullpage}
\usepackage{multicol,multirow}
\usepackage{tabularx}
\usepackage{graphicx}
\usepackage{float}
\usepackage{ulem}
\usepackage[utf8]{inputenc}
\usepackage[T2A]{fontenc}
\usepackage[russian]{babel}

% Оригиналный шаблон: http://k806.ru/dalabs/da-report-template-2012.tex

\begin{document}

\section*{Лабораторная работа №\,3 по курсу дискрeтного анализа: Профилирование
}

Выполнил студент группы 08-215 МАИ \textit{Тараскаев Давид}.

\subsection*{Дневник отладки}
Проведено нагрузочное тестирование на 10\,000\,000 операций с последующим анализом результатов через \texttt{gprof} и \texttt{valgrind}.

\subsection*{Инструмент \texttt{gprof}}
    \noindent\textbf{До оптимизации:}
    \begin{itemize}
      \item Функция \texttt{to\_lower} занимала \textbf{16,6\%} общего времени.
      \item Общее число вызовов — \textbf{13\,331\,642}.
      \item Причина: при каждом поиске ключа строка приводилась к нижнему регистру заново.
    \end{itemize}
    
    \noindent\textbf{После оптимизации:}
    \begin{itemize}
      \item Доля времени на \texttt{to\_lower} снизилась до \textbf{11,55\%}.
      \item Количество вызовов упало до ровно \textbf{10\,000\,000}, то есть количество вызовов стало равно количеству операций в тестовых данных.
      \item Оптимизация: преобразование ключа к нижнему регистру выполняется единожды перед вставкой, поиском или удалением; в узлах хранятся уже «нижнерегистровые» ключи.
    \end{itemize}

\subsection*{Инструмент \texttt{valgrind}}
\noindent\textbf{До оптимизации:}
\begin{itemize}
    \item Программа имела утечки по памяти.
    \item Причина: не был указан деструктор класса.
\end{itemize}

\noindent\textbf{После оптимизации:}
\begin{itemize}
    \item После добавления деструктора у класса, утечки по памяти пропали.
\end{itemize}

\subsection*{Тест производительности}

До оптимизации в реализованном AVL-дереве выполнение 10 миллионов случайных команд заняло 28.52 секунды, а после оптимизации — 28.35 секунды.
Разница во времени выполнения несущественна на данном объёме данных, однако при работе с большими объёмами она может стать более заметной.

\subsection*{Выводы}

% Описать область применения реализованного алгоритма. Указать типовые
% задачи, решаемые им. Оценить сложность программирования, кратко
% описать возникшие проблемы при решении задачи.
Профилировщики gprof и valgrind являются мощными и удобными инструментами для анализа производительности и выявления проблем в программе. Несмотря на простоту их использования, они выполняют важнейшую работу: позволяют обнаружить "узкие места", утечки памяти и избыточные вычисления, что особенно ценно при оптимизации сложных структур данных.
\end{document}