\documentclass[12pt]{article}

\usepackage{fullpage}
\usepackage{multicol,multirow}
\usepackage{tabularx}
\usepackage{graphicx}
\usepackage{float}
\usepackage{ulem}
\usepackage[utf8]{inputenc}
\usepackage[T2A]{fontenc}
\usepackage[russian]{babel}

\begin{document}

\section*{Лабораторная работа №\,4 по курсу дискрeтного анализа: Строковые алгоритмы}

Выполнил студент группы 08-215 МАИ \textit{Тараскаев Давид}.

\subsection*{Условие}

Необходимо реализовать один из стандартных алгоритмов поиска образцов для указанного алфавита.

\begin{itemize}
    \item Вариант алгоритма: Поиск одного образца при помощи алгоритма Бойера-Мура.
    \item Вариант алфавита: Слова не более 16 знаков латинского алфавита (регистронезависимые).
\end{itemize}

\subsection*{Описание программы}

\begin{description}
\item[{toLower()}] --- преобразует символы строки в нижний регистр
\item[{splitWords()}] --- разбивает строку на вектор слов с конвертацией регистра
\item[{buildBadChar()}] --- строит таблицу "плохих символов" для последнего вхождения слов
\item[{buildGoodSuffix()}] --- вычисляет оптимальные смещения для "хороших суффиксов"
\item[Основной алгоритм] --- комбинирует две эвристики для эффективного поиска:
\begin{itemize}
    \item Сравнение образца с текстом справа налево
    \item Выбор максимального смещения из двух эвристик
    \item Сохранение позиций всех найденных вхождений
\end{itemize}
\end{description}

\subsection*{Дневник отладки}


\subsection*{Тест производительности}

БОЙЕР-МУР - goodSuffix - Elapsed time: 3.96757 seconds
НАИВНЫЙ АЛГОРИТМ - Elapsed time: 44.3244 seconds

\subsection*{Выводы}


\end{document}
