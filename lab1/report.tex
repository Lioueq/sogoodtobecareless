\documentclass[12pt]{article}

\usepackage{fullpage}
\usepackage{multicol,multirow}
\usepackage{tabularx}
\usepackage{graphicx}
\usepackage{float}
\usepackage{ulem}
\usepackage[utf8]{inputenc}
\usepackage[russian]{babel}

% Оригиналный шаблон: http://k806.ru/dalabs/da-report-template-2012.tex

\begin{document}

\section*{Лабораторная работа №\,1 по курсу дискрeтного анализа: сортировка за линейное время}

Выполнил студент группы 08-215 МАИ \textit{Тараскаев Давид}.

\subsection*{Условие}


Требуется разработать программу, осуществляющую ввод пар «ключ-значение», их упорядочивание по возрастанию ключа указанным алгоритмом сортировки за линейное время и вывод отсортированной последовательности.\\
Метод сортировки: поразрядная сортировка.\\
Тип ключа: MD5-суммы (32-разрядные шестнадцатиричные числа).\\
Тип значения: строки фиксированной длины 64 символа, во входных данных могут встретиться строки меньшей длины, при этом строка дополняется до 64-х нулевыми символами, которые не выводятся на экран.


\subsection*{Метод решения}

На вход прнинимаются пары ключ-значение и передаются в вектор. После окончания ввода пар, сортируем вектор с помощью алогритма поразрядной сортировки. После окончания сортировки, пары из вектора выводятся в отсортированном порядке в формате key  value.

\subsection*{Описание программы}

Anivector - структура, которая реализовывает вектор.\\
PairKV - структура, которая реализовывает пару ключ - значение.\\
RadixSort - функция, которая реализовывает алгоритм поразрядной сортировки.

\subsection*{Дневник отладки}

Изначально написал код, ни чем себя не ограничивая, потом заменял запрещенные структуры на разрешенные. Могу только выделить свою 6 и 7 посылки. В 6 посылке использовалось только то, что разрешено, но программа получила ML. Оптимизация алгоритма сортировки решила проблему.

\subsection*{Тест производительности}

\begin{figure}[H]
    \centering
    \includegraphics[width=0.8\linewidth]{image1.png}
    \caption{График зависимости времени работы программы от N}
    \label{fig:graph}
\end{figure}

\subsection*{Выводы}

% Описать область применения реализованного алгоритма. Указать типовые
% задачи, решаемые им. Оценить сложность программирования, кратко
% описать возникшие проблемы при решении задачи.
Так как сложность алгоритма O(N*K), то такую сортировку используют для сортировки больших данных, например: базы данных, логи. Также используют для алгоритмов сжатия.

\end{document}